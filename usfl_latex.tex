\documentclass[11pt, a4paper]{book}
\usepackage{physics}
\usepackage[margin=1in, headheight=14pt]{geometry}
\usepackage{amsfonts,amsmath,amssymb,suetterl}
\usepackage{lmodern}
\usepackage[T1]{fontenc}
\usepackage{mdwlist}
\usepackage[nodisplayskipstretch]{setspace}
\usepackage{float,graphicx}
% \usepackage{tikz}
\usepackage{pgfplots}

\pgfplotsset{compat = newest}
\usetikzlibrary{arrows.meta}

\setstretch{1.5}
\pagestyle{plain}

\newcommand{\lmrk}[2]{\hfill\textbf{Total for Question #1 is #2 marks}}
\newcommand{\mrk}[1]{\hfill\textbf{(#1)}}
\newcommand{\strch}{\vspace{\stretch{1}}}
\newcommand{\dline}{\tikz\draw[thick, dashed] (0,0) -- (3,0);}

\parindent 0ex
\setlength{\parskip}{1em}

\begin{document}
  \begin{figure}[H]
    \centering
    \begin{tikzpicture}
      \begin{axis}[
          xmin = -1, xmax = 10,
          ymin = -2, ymax = 30,
          xticklabels = {},
          yticklabels = {},
          % xtick distance = 2.5,
          % ytick distance = 0.5,
          % grid = both,
          % minor tick num = 1,
          % major grid style = {lightgray},
          % minor grid style = {lightgray!25},
          axis lines = middle,
          % width = \textwidth,
          % height = 0.5\textwidth,
          xlabel = {$x$},
          ylabel = {$y$},
        ]
        % Plot a function
        \addplot[
          domain = -0.8:8.8,
          samples = 200,
          smooth,
          thick,
        ] {x^2 - 8*x + 21};
        \node at (7.5, 28) {$y = f(x)$};
        \node at (-0.4, -1) {$O$};
        \node at (4, 3) {$M$};
        \addplot[
          mark=x,
          mark size=6pt
        ]
        coordinates {
          (4, 5)
        };
      \end{axis}
    \end{tikzpicture}
  \end{figure}
\end{document}