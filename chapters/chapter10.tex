\chapter{GCSE Revision - Straight Line Equations}

\begin{enumerate}
  \item Finding a gradient
  \begin{enumerate}
    \item What is the gradient of the line that goes through the points $(1,6)$ and $(5,-3)$.\strch
    \item What is the gradient of the line $x + 2y = 1$?\strch
  \end{enumerate}
  \item Finding the equation of the line given two points.
  \begin{enumerate}
    \item Give the full equation of the line which goes through the points $(3,5)$ and $(5,11)$.\strch
    \item Give the full equation of the line which goes through the points $(5,1)$ and $(8,-8)$.\strch
    \item Give the full equation of the line which has the gradient 4 and goes through the point $(0,3)$.\strch
    \item Give the equation of the line which has gradient 4 and goes through the point $(3,7)$.\strch
  \end{enumerate}
  \newpage
  \item Finding the equation of a line parallel or perpendicular to another.
  \begin{enumerate}
    \item Give the equation of the line which is parallel to $y = 4x + 3$ and goes through the point $(4,5)$.\strch
    \item Give the equation of the line which is parallel to $y = \frac{1}{3} x - 2$ and goes through the point $(9,5)$.\strch
    \item Give the equation of a line which is perpendicular to $y = 2x + 1$.\strch
    \item Give the equation of the line which is perpendicular to $y = 5x + 6$ and goes through the point $(-15,2)$.\strch
  \end{enumerate}
  \item Finding where a line intercepts the $x$ or $y$ axis.
  \begin{enumerate}
    \item The y-axis:\strch
    \item The x-axis:\strch
  \end{enumerate}
  \newpage
  \item At what point does $y = 3x - 2$ intercept:
  \begin{enumerate}
    \item The y-axis: \strch
    \item The x-axis: \strch
  \end{enumerate}
  \item A and B are straight lines. Line A has equation $2y = 3x + 8$. Line B goes through the points $(-1, 2)$ and $(2, 8)$. Do lines A and B intersect? You must show all your working.\mrk{3}\strch
  \item \mbox{}
  \begin{figure}[H]
    \centering
    \begin{tikzpicture}[>=latex, scale=0.8]
      \tkzDefPoints{-1/0/X1, 6/0/X2, 0/-2.5/Y1, 0/6/Y2, 0/-1.5/P, 5/2/B}

      \tkzDrawSegment[->](X1,X2)
      \tkzDrawSegment[->](Y1,Y2)

      \tkzInterLL(P,B)(X1,X2)
      \tkzGetPoint{A}

      \tkzDefLine[perpendicular=through A](P,B)
      \tkzGetPoint{d}

      \tkzInterLL(A,d)(Y1,Y2)
      \tkzGetPoint{D}

      \tkzDefLine[perpendicular=through D](D,A)
      \tkzGetPoint{c1}

      \tkzDefLine[perpendicular=through B](A,B)
      \tkzGetPoint{c2}

      \tkzInterLL(B,c2)(D,c1)
      \tkzGetPoint{C}

      \tkzDrawSegments(P,B A,D D,C B,C)
      \tkzMarkRightAngles(D,A,B A,B,C B,C,D C,D,A)

      \tkzLabelPoints[below left](P)
      \tkzLabelPoints[below](A)
      \tkzLabelPoints[right](B)
      \tkzLabelPoints[above](C)
      \tkzLabelPoints[left](D)
    \end{tikzpicture}
  \end{figure}
  $ABCD$ is a square. $P$ and $D$ are points on the $y$-axis. $A$ is a point on the $x$-axis. $PAB$ is a straight line.\\
  The equation of the line that passes through the points $A$ and $D$ is $y = -2x + 6$. Find the length of $PD$.\mrk{4}\strch
  \newpage
  \item \mbox{}
  \begin{figure}[H]
    \centering
    \begin{tikzpicture}[>=latex, scale=0.8]
      \tkzDefPoints{-4/0/X1, 4/0/X2, 0/-4/Y1, 0/4/Y2, -0.6/1/A, 3/3.5/B, 7/3/L}

      \tkzDrawSegment[->](X1,X2)
      \tkzDrawSegment[->](Y1,Y2)

      \tkzDrawPoints[shape=cross out, size=4pt](A,B)

      \tkzLabelPoint[above left](A){A(-1,2)}
      \tkzLabelPoint[above right](B){B(7,5)}

      \tkzLabelPoint[above](L){Diagram \textbf{NOT}}
        \tkzLabelPoint[below](L){accurately drawn}
    \end{tikzpicture}
  \end{figure}
  $A$ is the point $(-1, 2)$. $B$ is the point $(7, 5)$.
  \begin{enumerate}
    \item Find the coordinates of the midpoint of $AB$.\strch \\\vspace*{0pt}\hfill(\dline,\dline)
    \item a
  \end{enumerate}



\end{enumerate}

